\documentclass[12pt,letterpaper]{article}
%\documentclass[12pt,a4paper,fleqn]{article}
\usepackage{graphicx}
\usepackage{epsfig}
\sloppy
%\flushbottom

%\pagestyle{empty}

\title{BCI2000 Command Line Interface}
%\date{}
\author{J\"{u}rgen Mellinger}

\begin{document}
\maketitle
\tableofcontents

\pagebreak
\section{Introduction}

Command line interfaces are a powerful tool for dynamically combining 
pieces of software which are typically tiny programs optimized for performing
a very limited functionality.
With a command line interface, data processing is usually performed 
\textit{sequentially} on a stream of data.
This allows the processing of indefinitely large amounts of data while 
consuming but a finite, and often very small, amount of memory.

Moreover, combining command line programs to perform complex tasks can be 
done interactively by entering complex commands directly but also in a ``scripting''
manner, by entering sequences of commands into text files which allow for 
iteration and branching. 
That way, automation of complex tasks can be achieved by rather simple means.

With the BCI2000 command line interface, BCI2000 data files can be 
\begin{itemize}
\item \textit{converted} into various formats, ready for
further processing with available software tools, or for visual inspection in 
human readable format; \item \textit{processed} off-line with literally the same
code as used in the on-line system.
\end{itemize}

Together, these two concepts provide versatile access to recorded data on all 
stages of processing.
The BCI2000 command line interface may prove useful for
\begin{itemize}
\item automated \textit{analysis} of large amounts of recorded data,
\item \textit{development} of new filter classes,
\item \textit{verification} of external off-line analysis methods.
\end{itemize}

This document explains the command line tools available for working with BCI2000 
files, and how to build BCI2000 filters as single executables that may be combined 
to form a chain of filters.

\section{Command line interfaces}

Basically, the interface of a command line tool consists in 
\begin{itemize}
\item an input stream \texttt{stdin},
\item an output stream \texttt{stdout},
\item an error stream \texttt{stderr},
\item optional arguments (switches) controlling details of its behavior.
\end{itemize}

The most important feature of a command line interface is its ability to 
\textit{redirect} streams.
\begin{itemize}
\item \texttt{stdin} can be redirected to read from a file using the 
      \texttt{<} operator;
\item \texttt{stdout} can be redirected to write to a file using the 
      \texttt{>} operator;
\item one program's \texttt{stdout} can be plugged into another program's 
      \texttt{stdin} using the \texttt{|} operator (``pipe'').
\end{itemize}

Because redirection is a feature of the program's execution environment 
(``shell''), and not a feature of the program itself, programs can be very 
simple, avoiding handling of file names and associated I/O errors.

The following example command line will extract the ``sampling rate'' 
parameter from the BCI2000 data file \texttt{mydata.dat}, and display 
the result on the text console:
\begin{verbatim}
bci_dat2stream < mydata.dat | bci_stream2prm | grep SamplingRate
\end{verbatim}

\section{Building BCI2000 command line tools}
As the command line tools are not automatically built from the main makefile, 
you need to do this manually. 
On the Windows NT shell, change to the \texttt{BCI2000/src/Tools/cmdline} directory,
and execute \texttt{make install} from there; this will also move the 
resulting executable to the \texttt{BCI2000/Tools/cmdline} directory.
To build an executable from a file \texttt{MyFilter.cpp} containing a 
BCI2000 filter \texttt{MyFilter}, execute \texttt{MyFilter.install}.
This will build an executable \texttt{MyFilter.exe} and add it to the
\texttt{BCI2000/Tools/cmdline} directory.

Building command line tools does not require the full Borland C++ Builder 
IDE; you may use the freely available command line version of the Borland C++ compiler
instead (see section \ref{sec:furtherinfo} for information on how to obtain it).
In this case, you will need to out-comment the line reading
\verb|#include <vcl.h>| 
from the file \texttt{BCI2000/src/shared/PCHIncludes.h}
(thanks to Roberta Carabalona for pointing this out).

In order to use the command line tools, you will need to add the full path 
to the \texttt{BCI2000/Tools/cmdline} directory to your system's PATH environment variable. 
Please consult your operating system's documentation on how to do this.

\section{Stream format}
The common format used to exchange data between BCI2000 command line tools 
is called a ``BCI2000 binary stream,'' or just ``stream.''
Such a ``stream'' transfers parameters, states, and data in exactly the same
binary format as it is used for socket communication between the four main 
modules of the BCI2000 real-time system.

The stream format is different from the BCI2000 \texttt{dat} and \texttt{prm} 
file formats, and is not human readable.
A number of tools is available to convert data either to or from stream format.
As the stream format is only used between tools, and generally not of interest 
to the user, a typical conversion will use \textit{two} of the tools provided, 
one to translate the original file into a ``stream,'' and one to translate the 
stream into the desired format.

E.g., to display a \texttt{dat} file's contents in a human readable format,
one would use a tool called \texttt{bci\_dat2stream} to convert it into a stream, 
and then use another tool called \texttt{bci\_stream2asc} to convert the stream 
into text. 
If we want the output to appear in the text window as a sequence of pages, we pipe it
into the ``more'' program:
\begin{verbatim}
bci_dat2stream < myfile.dat | bci_stream2asc | more
\end{verbatim}

\subsection{Conversion into stream format}
\subsubsection{\texttt{bci\_dat2stream}}
\paragraph{Input format} \texttt{bci\_dat2stream} converts a BCI2000 data  
(\texttt{dat}) file into a stream.
\paragraph{Options} The \texttt{--transmit} option may be used to select states, 
parameters, and data for transmission, as in\\
\verb|bci_dat2stream --transmit-sp < myfile.dat|\\
In the above example, only states and parameters but no data will be contained in 
the resulting stream.
``Data'' comprises signal and state vector data.
By default, all information contained in the \texttt{dat} file will be transmitted,
corresponding to \texttt{--transmit-spd}.

\subsubsection{\texttt{bci\_prm2stream}}
\texttt{bci\_prm2stream} converts a BCI2000 parameter (\texttt{prm}) file into a stream.

\subsection{Conversion from stream format}
\subsubsection{\texttt{bci\_stream2prm}}
\texttt{bci\_stream2prm} converts a stream into a BCI2000 parameter (\texttt{prm}) file.
As a parameter file is just a sequence of parameter lines, this is also a text-only, 
human readable format.

\subsubsection{\texttt{bci\_stream2asc}}
\texttt{bci\_stream2asc} converts a stream into a human readable format.�
Each object contained in the stream will appear as its C++ type name, followed by an 
opening brace, its content, and a closing brace.
The content will appear as defined by the stream inserter \texttt{operator>>} for 
the object's type.

In the output of BCI2000 state vector information, each state will appear on its 
own line, thus values of certain states may be easily extracted using the 
\texttt{grep} program.

\subsubsection{\texttt{bci\_stream2table}}
\texttt{bci\_stream2table} converts a stream into a tab-separated table containing 
state and signal values in ASCII format.
Each state, and each entry of the signal, has its own column.
The first line of output begins with a \texttt{\#} comment character, and contains 
a tab-separated list of column headers.
This format is best suited for data import into applications that work on tables.

\subsubsection{\texttt{bci\_stream2mat}}
\texttt{bci\_stream2mat} converts a stream into a Matlab binary file.
The output \texttt{.mat} file contains two Matlab variables called
\texttt{Index} and \texttt{Data}.
Of these, the \texttt{Data} variable is a matrix with each column representing a
BCI2000 data block (comprising state information and signal data).
\texttt{Index} is a Matlab structure that contains indices to \texttt{Data's}
rows, allowing access to BCI2000 states by name, as in:

\verb|myMatlabVariable = squeeze( Data( Index.TargetCode, : ) );|\\
As each BCI2000 data block contains a signal which is a two-dimensional matrix 
(channels by elements), the signal index is itself a matrix.
To copy the first channel's data into a Matlab variable, write

\verb|myChannel1 = squeeze( Data( Index.Signal( 1, : ), : ) );|\\

For convenience, there is a Matlab function provided that simplifies reading 
\texttt{bci\_stream2mat} output files into Matlab variables:
\begin{verbatim}
[ mySignal, myTargetCode ] = 
  load_bcimat( 'eegdata.mat', 2, 'TargetCode' );
\end{verbatim}
This function takes the file name as its first argument.
In a second argument, specify the number of dimensions your output signal will
have -- typically, this will be 2 for EEG-like data (samples by channels), and
3 for spectral data (blocks by bins by channels).
Remaining arguments are treated as state names; the associated state data will be 
written into the variables specified as remaining output arguments.
State variables will always be one-dimensional, with their number of entries matching
the first dimension of the signal variable.

\section{Applying BCI2000 filters to streams}
\subsection{Compiling an existing filter as a command line tool}
A filter defined in a file \texttt{MyFilter.cpp} may be compiled and linked into 
its own executable by executing\\
\verb|make MyFilter.exe|\\
from the Windows command prompt when in \texttt{BCI2000/Tools/cmdline}.
For this to work, it is necessary that the directory containing the filter's 
\texttt{cpp} file is contained in the makefile's \texttt{SIGPROC} variable; you may 
have to adapt it to fit your needs.

If the filter's code depends on code not contained within its \texttt{cpp} file,
you will get linker complaints about unresolved externals.
To solve the problem, add the missing \texttt{cpp} files to the makefile's 
\texttt{BCIOBJ} variable, and execute\\
\verb|make clean && make MyFilter.exe|\\
To check whether the executable works, enter \\
\verb|MyFilter --help|\\
You should get a message that the program applies the ``MyFilter'' filter to its input.

\subsection{Off-line only filters}
For off-line analysis, data must often be partitioned into ``segments'' before
performing statistics. As there is no notion of ``segments'' in the on-line data
and file format, we suggest using the ``Running'' state to indicate segments in
the following way:

When performing segmenting, a filter sets the ``Running'' state to zero
outside segments.
A statistics filter will then perform buffering from its \texttt{Process()}
function, and act on the buffered data from its \texttt{StartRun()} and 
\texttt{StopRun()} functions.

Setting the ``Running'' state to zero will suspend the on-line system, so
this kind of segmenting and statistics filtering cannot be used on-line.

\section{Examples}
The following examples work from the Windows NT command prompt.
Nevertheless, we would like to point the reader to the free \textit{cygwin} 
collection of GNU tools ported to the Win32 API. \textit{Cygwin} provides the 
power of the \texttt{bash} shell, and of programs like the \texttt{sed} 
stream editor. 
Although the BCI2000 command line tools cannot be compiled with \texttt{gcc} they 
work fine when called from \textit{cygwin}.

\subsection{Extracting parameters}
To extract parameters from a data file, convert it into a stream using 
\texttt{bci\_dat2stream --transmit-p}, and convert the stream into a parameter file
using \texttt{bci\_stream2prm} as in
\begin{verbatim}
bci_dat2stream < mydata.dat | bci_stream2prm > myprms.prm
\end{verbatim}

\subsection{Processing data with parameters different from the ones 
contained in the file}
To combine a data file with parameters other than those contained in it, use 
\texttt{bci\_dat2stream}'s \texttt{--transmit} option to suppress parameters, 
and combine its output with \texttt{bci\_prm2stream}'s into a single stream. 
To affect processing, parameters must precede the data in the stream.
Combining the output is effected by the \texttt{( ... \&\& ... )} construct.
\begin{verbatim}
(bci_prm2stream < myparameters.prm && 
  bci_dat2stream --transmit-sd < mydata.dat) |
  MyFilter | bci_stream2table > mytable.txt
\end{verbatim}

\subsection{Processing data with BCI2000 filters, and importing the
results into Matlab}
To process data with the filters used in the mu-training on-line system, 
saving the AR spectrum as Matlab file, execute
\begin{verbatim}
bci_dat2stream < mydata.dat | TransmissionFilter | 
  CalibrationFilter | SpatialFilter | ARFilter | 
  bci_stream2mat > myspectra.mat
\end{verbatim}
Load the data into Matlab using
\begin{verbatim}
[ signal, TargetCode ] = 
  load_bcimat( 'myspectra.mat', 3, 'TargetCode' );
\end{verbatim}
This requires that the file \texttt{load\_bcimat.m} is accessible from
the Matlab search path.

\subsection{Exporting BCI2000 data into a table suitable for import into 
other applications (MS Excel, SPSS)}
To process data with the filters used in the mu-training on-line system, 
saving the AR spectrum as a table in ASCII format, execute
\begin{verbatim}
bci_dat2stream < mydata.dat | TransmissionFilter | 
  CalibrationFilter | SpatialFilter | ARFilter | 
  bci_stream2table > mytable.txt
\end{verbatim}

\subsection{Testing a modified filter on existing data}
To verify that changes to a filter's code don't change its behavior with 
respect to existing data, apply both versions to a stream, convert the output 
stream into human readable format, and have a file comparison program display 
any differences.
For the following example, we will compare a previous version of the ARFilter, 
renamed \texttt{prev\_ARFilter}, to the current one.
\begin{enumerate}
\item Create a stream suitable for input to the ARFilter:
\begin{verbatim}
bci_dat2stream < mydata.dat | TransmissionFilter | 
  CalibrationFilter | SpatialFilter > test.bcistream
\end{verbatim}
\item Apply both filter versions to the stream, and save the results in human readable format:
\begin{verbatim}
ARFilter < test.bcistream | bci_stream2asc > ARresult.txt &&
  prev_ARFilter < test.bcistream |
  bci_stream2asc > prev_ARresult.txt
\end{verbatim}
\item Compare the results (using the Windows NT analog to the \texttt{diff} program):
\begin{verbatim}
comp /a /l prev_ARresult.txt ARresult.txt | more
\end{verbatim}
\end{enumerate}

\section{Further information}
\label{sec:furtherinfo}
\begin{flushleft}
\begin{itemize}
\item A comprehensive and up-to-date description of a number of command line shells, and
their scripting, is provided at \texttt{http://www.ss64.com/}.
\item The \textit{cygwin} Win32 port of GNU tools can be downloaded at
\texttt{http://www.cygwin.com/}.
\item The free command line version of the Borland C++ compiler is available at 
\texttt{http://www.borland.com/downloads/download\_cbuilder.html}.
\end{itemize}
\end{flushleft}


\end{document}
