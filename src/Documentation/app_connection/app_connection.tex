\documentclass[12pt,a4paper,notitlepage]{article}
%\documentclass[12pt,a4paper,fleqn]{article}
\usepackage{graphicx}
\usepackage{epsfig}
\sloppy
\flushbottom

%\pagestyle{empty}

\title{BCI2000 External Application Interface}
%\date{}
\author{J\"{u}rgen Mellinger}

\begin{document}
\maketitle
\tableofcontents

\pagebreak
\section{Introduction}

The BCI2000 external application interface provides a bi-directional link to exchange 
information with external processes running on the same machine, or on a different machine
over a local network.

Its design aims at simplicity, and at minimal interference with the timing of the signal flow
through the BCI2000 system.
With this in mind, a connection-less, UDP based transmission protocol was chosen rather than a TCP based one.
This comes at the cost of a possible loss, or re-ordering of protocol messages.
To keep the probability for such losses as low as possible, and their consequences as
local as possible, messages have been designed to be short, self-contained, and redundantly
encoded in a human readable fashion.

For communication involving a number of network nodes, or unreliable connections, we suggest 
using locally executed server processes that forward messages to a TCP connection.

\section{Description}

For each block of data processed by the BCI2000 system, two types of information are sent out
and may be received from the external application interface:
\begin{itemize}
\item the BCI2000 internal state as defined by the values of the entries of its state vector data structure, and
\item the BCI2000 control signal.
\end{itemize}
For the definition of and details about states and the control signal, please refer to the BCI2000
project outline document.

Sending data is done after processing by the application module's task filter has taken place; receiving occurs before the task filter.
This makes sure that changes resulting from user choices are sent out immediately, and that 
received information will immediately be availiable to the task filter.

IP addresses and ports used are user-configurable. Sending and receiving need not use the same address and port.


\section{Protocol}

Messages consist of a name and a value, separated by white space
and terminated with a single newline \verb|\n| character.

Names may identify
\begin{itemize}
\item BCI2000 states by name -- then followed
   by an integer value in decimal ASCII representation;
\item Signal elements in the form \texttt{Signal(<channel>,<element>)} --
   then followed by a float value in decimal ASCII representation.
\end{itemize}

\paragraph{Examples:}
\begin{verbatim}
Running 0\n
ResultCode 2\n
Signal(1,2) 1e-8\n
\end{verbatim}

Note that the first example will switch BCI2000 into a suspended state.
While the system is in that state, no communication is possible over the application protocol.

\section{Parameterization from within BCI2000}

BCI2000 reads data from a local IP socket specified in the \texttt{ConnectorInputAddress} parameter, and writes data out into the socket specified in the
\texttt{ConnectorOutputAddress} parameter.
Sockets are specified by an address/port combination.
Addresses may be host names, or numerical IP addresses. Address and port are separated by
a colon as in
\begin{verbatim}
localhost:5000
134.2.103.151:20321
\end{verbatim}

For incoming values, messages are filtered by name using a list of allowed names 
present in the \texttt{ConnectorInputFilter} parameter. 
To allow signal messages, allowed signal elements must be specified including their
indices.
To allow all names, enter an asterisk (*) as the only list entry.

\end{document}
