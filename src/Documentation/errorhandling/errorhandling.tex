\documentclass[12pt,a4paper]{article}
%\documentclass[12pt,a4paper,fleqn]{article}
%\usepackage{ngerman}
%\usepackage{bibgerm}
\usepackage{graphicx}
\usepackage{epsfig}
%\sloppy
%\flushbottom

%\pagestyle{empty}

\title{BCI 2000 Error Handling: A Proposal}
%\date{}
\author{J\"urgen Mellinger, Gerwin Schalk}

\begin{document}
\maketitle
\tableofcontents

\pagebreak
\section{Handling Errors}
\label{secHandl}
\subsection{Types of Errors}
We assume that all errors we need to consider fall 
into one the following categories, each of which
implies a different type of approach to error
avoidance/error handling:
\begin{itemize}
\item{Parameter Setup Errors}
\item{Runtime Errors}
\item{Logic (Programming) Errors}
\end{itemize}

\subsection{Parameter Setup Errors}

\subsubsection{Definition of the Term}

This category covers anything a user can do wrong
by using a program with parameters that are out of range,
inconsistent, or otherwise erroneous (by, e.g.,
specifying an output file at a location where the user
has no write permission).

Parameter setup errors, when unhandled, become runtime errors.

\subsubsection{Proposed Strategies}
\label{secParamcheck}

As a guideline for approaching Parameter Setup Errors
we propose the adoption of the following principle: ``Whatever a
user does from within an application program should never
make that application crash.''

In BCI 2000, this translates into a thorough parameter
check done by each module before any parameter settings are
actually applied to the system.

Parameter checking should comprise
\begin{itemize}
  \item{Range and Consistency checks,}
        whereby generally ranges depend on other parameters' values;
  \item{Signal property checks:}
        Does the output signal of one filter meet the next filter's
        requirements for its input signal?
  \item{Resource availability checks:}
  \begin{itemize}  
    \item{Are needed system resources available?}
          (E.g., is it possible to open a required sound output device?)
    \item{Are auxiliary files (e.g., media files) available and readable?}
    \item{Do output files have legal file names? Are output files writeable?}
       (We could even check whether there is enough
       space left to write the EEG file, but practically this
       would not make too much sense because a concurrent process
       might use up the space while our system runs.)
  \end{itemize}
\end{itemize}

In each of those cases, the user should get appropriate
feedback guiding her towards fixing the problem.

Whenever the system tries to fix a parameter setup error
by using some default set of parameters, it should do so
only if
\begin{itemize}
\item it presents the user with a warning that tells her
what it did and why it did so, and if
\item the automatically fixed parameters are treated as if
changed by the user, i.e. with a parameter check performed
on them.
\end{itemize}
Otherwise, people might end up using a system that doesn't
do what they want it to -- but without telling them, so they
don't ever realize --, or with a system creating new parameter
inconsistencies when trying to fix others.

\subsubsection{User Interface Details}

The user interface for Parameter Setup Error handling is,
along with the parameter setup dialog, part of the operator
module. A first implementation of a GUI based user interface might
consist in a floating, non-modal error window popping up from the operator
module that would present a list of error related textual
messages to the user, allowing for browsing error messages
while changing respective parameters via the parameter setup
dialog. After the next parameter check, the operator module would close
that window or replace its contents based on the result of the check.
Parameter checking would occur when the user clicks the "SetConfig"
button in the operator main window, followed by actually applying
parameters in case the check was successful.

\subsection{Runtime Errors}

\subsubsection{Definition of the Term}

This category covers everything that can go wrong
in the course of an application program running
insofar as that malfunction is due to a lack of resources in the
underlying system required for proper operation
(i.e., not due to a programming error).
Assuming that parameter checking has been implemented
properly as outlined above, we can narrow the term
'Runtime Error' to cases for which the following statement
holds: A runtime error occurs whenever the system runs
out of resources that were still available during parameter
checking.

\ 

Typical reasons for this kind of error are
\begin{itemize}
\item the system running out of disk space while recording data,
\item files being moved, trashed, or locked by a concurrent process,
\item a network connection becoming unavailable.
\end{itemize}

Runtime errors, when unhandled, become logic errors because
the code implies assumptions that no longer hold once a
runtime error has occurred.

\subsubsection{Proposed Strategies}

In a properly designed and implemented system, runtime errors
in the restricted sense described above will not occur 
frequently. However, as they are caused by undesired
circumstances outside the scope of the application program
itself, it seems important to provide information
to the user as detailed as possible in order to enable her
to prevent this type of situation in the future, and to 
make her aware of the fact that the application program
depends on her willingness to provide a smooth operating
environment. This being ensured, it seems appropriate to
simply abort execution altogether, while trying to avoid 
a loss of the data acquired up to that time.

\subsubsection{User Interface Details}

In general, it is desirable to have runtime errors displayed
along with the operator module's user interface. However,
as this requires a working connection between the module where
the error occurs, and the operator module, this may not
always be possible. Therefore, in addition to an
operator-based error reporting interface, each module should
have a less demanding mechanism to provide error information to the
user, e.g., a local log file.

\subsection{Logic Errors}

\subsubsection{Definition of the Term}

Logic, or programming, errors in general are faults of
a programmer who, in his or her code, implicitly or explicitly
makes assumptions that do not always hold.

\subsubsection{Proposed Strategies}

Programming errors are not supposed to occur at all in
a tested version of an application. Therefore, instead
of trying to 'handle' them, it is important to make them show
up as close to their point of origin in the code as
possible, by frequently and explicitly checking whether
implicit assumptions actually hold, and aborting execution
with an error message if this is not the case.

Aside from that, writing code as explicit, general, and
simple as possible greatly reduces the possibility of making
logic errors in the first place.

\subsubsection{User Interface Details}

As programming errors are nothing a user can do anything
about, and as their occurrence with a user is an embarrassing
glitch anyway, simply aborting the program or module with
an error message seems appropriate.

\pagebreak
\section{Implementation Details}

\subsection{Interface to the Programmer}

\subsubsection{Reporting Errors}

For a simple and general way to provide
user communication and error reporting means to a module's
programmer, we propose the introduction of two global
objects derived from \texttt{std::ostream} and named, e.g.,
\texttt{bciout} and \texttt{bcierr}, in analogy to
\texttt{std::cout} and \texttt{std::cerr}, where \texttt{bciout}
is used to transfer general messages and warnings while
\texttt{bcierr} takes actual errors. \footnote{The introduction
of error numbers seems not necessary at this point;
however, if error codes are preferred, they should
be implemented as an enumerated type, not spelled
out explicitly in strings.}

A code example would then look like this:
\begin{verbatim}
...
if( myParamValue < 0 || myParamValue >= 10 )
{
  bcierr << "'myParam' is out of range." << endl;
  errorOccurred = true;
}
...
\end{verbatim}

Furthermore, for handling runtime errors difficult to recover
from, a programmer may also throw an exception that will abort
execution and
eventually lead to an error message being sent to the operator
module (for framework related details see section~\ref{secImpl}):
\begin{verbatim}
...
if( ernie.find( bert ) != ernie.end() )
  throw "Ernie just ate Bert. " 
        "I don't know how to tell the story.";
tellMyStory( bert.begin(), ernie.end() );
...
\end{verbatim}

\subsubsection{Checking Parameters}

Checking parameters is done in a separate member function
of the filter base class which, similar to the member function
that does the actual processing, takes input and output signal
representatives as parameters, thus allowing for signal
property checking.

For an actual implementation, its declaration would be as follows:
\begin{verbatim}
void GenericFilter::Preflight(
   const SignalProperties* inSignalProperties,
         SignalProperties* outSignalProperties ) const;
\end{verbatim}

For a filter class derived from
\texttt{GenericFilter}, this function is supposed to perform
parameter checking as described in section~\ref{secParamcheck}.
Instead of returning an error value, it writes possible error
messages into \texttt{bcierr}.
The \texttt{const} declaration for its \texttt{this} pointer
prohibits initialization functionality from
\texttt{GenericFilter::Initialize()} entering into
\texttt{Preflight()}; this is unwanted because it would
corrupt the idea of performing a \textit{complete} parameter check
before actually altering the state of \textit{any}
filter object.

\subsection{Implementation on the Framework Side}
\label{secImpl}

The operator module's behaviour in response to an error
message arriving from one of the modules would depend
on its context, i.e. on the execution phase the system is in.
That way, no additional programming interface elements
visible to a filter/module programmer
would be needed to implement an error handling scheme as
described in section~\ref{secHandl}. 

During the \textbf{preflight phase,} errors would be {Parameter
Setup Errors.} A module's framework code behind
\texttt{bcierr} would just collect error
messages; on return from the preflight function, it would
send those messages to the operator module which would then,
from the contents of the message (i.e. whether it was empty
or not), determine whether the preflight was successful;
on not receiving any message after some timeout\footnote{
For now, a simple timeout scheme with a fixed timeout
interval of 5~s seems appropriate. In the future, one might consider
a module requesting additional timeout periods if it expects 
lengthy calculations.}
it would assume a broken connection or a crashed module.

During all \textbf{other phases,} the code behind \texttt{bcierr}
would immediately (i.e., on flushing the \texttt{std::ostream})
send its message buffer to a log file as well as to
the operator module, indicating a {Runtime Error} 
to the operator module which would, in turn, halt the system,
shut down the other modules, and display the message to the
user.

In addition, the top level exception handling code
of each module would contain similar functionality,
sending an exception's associated description string into
a log file and to the operator module, if possible, then
quitting the module in which the exception occurred.
This would not only ensure
a proper general handling of exceptions within the framework but
also allow a programmer to handle {Runtime Errors} by raising
her own exceptions, eliminating the need to take care of the error
condition in the code following the detection of an error.

%\subsection{Appendix: Using the \texttt{assert()} Macro
%to deal with Logic errors}
%
%We propose using the \texttt{assert()} macro as a valuable
%tool to make implicit assumptions explicit, as in the
%following examples:
%\begin{verbatim}
%#include <assert>
%...
%
%void DoSomething( int someArg )
%{
%  // We implicitly assume someArg to be smaller that 10.
%  // So, let's make that assumption explicit and be sure
%  // we'll notice whenever it is violated.
%  assert( someArg < 10 );
%  ...
%}
%
%void DoSomethingElse( someEnumeratedType someArg )
%{
%  switch( someArg )
%  {
%    case firstExpectedValue:
%      DoWhatever1();
%      break;
%
%    case secondExpectedValue:
%      DoWhatever2();
%      break;
%
%    // Catch all unexpected cases.
%    default:
%      // We could just write assert( false ) but the following
%      // trick gives us a more useful error message.
%      assert( !"Unexpected value of someArg\n" );
%  }
%}
%
%float ComputeSomeFunction( float x )
%{
%  assert( x != 0.0 );
%  return 1.0 / x * ( pow( x, 2 ) + x );
%}
%
%// Or, even better because more general and more simple
%// (sorry for the stupid example):
%float ComputeSomeFunction( float x )
%{
%  // No implicit assumption, no assert().
%  // And a more precise result, too.
%  return x + 1.0;
%}
%\end{verbatim}
%
%Whenever the argument of \texttt{assert()} becomes \texttt{false},
%the program will be aborted with an error message. When the code
%is compiled with the symbol \texttt{NDEBUG} being \texttt{\#define}d
%via a compiler option, all occurrences of \texttt{assert()} and its
%arguments will be ignored.
%
%\textbf{Caveat:} Don't use any function calls
%with side effects inside \texttt{assert()} because they won't be
%called in the final \texttt{NDEBUG} version! 

\pagebreak
\section{Error Handling Roadmap}
\begin{flushleft}
\begin{itemize}
\item \textit{Mellinger:} Remove references to current preliminary error handling
object from code. \textit{done Oct 21, 2002}
\item \textit{Mellinger:} Introduce \texttt{GenericFilter} inheritance into all
existing filter classes.
\item \textit{Mellinger:} Create headers and dummy implementation for error stream
objects in framework code to allow for using error handling from
filter code; add virtual function \texttt{GenericFilter::Preflight}
and a dummy implementation to allow for existing code to compile.
\item \textit{Mellinger:} Introduce calls to \texttt{GenericFilter::Preflight} into
framework code.
\item \textit{Mellinger:} Implement display of error messages in operator module.
\item \textit{Mellinger:} Write actual implementation for error stream objects that
sends errors to operator.
\item \textit{Existing filters' authors:} Add implementation of \texttt{GenericFilter::Preflight} to
existing filters.
\item \textit{Mellinger:} Write top level exception handler that diverts exceptions
occurring in modules to the operator module and remove \texttt{GenericFilter::Preflight} dummy default implementation from
\texttt{GenericFilter.}
\end{itemize}
\end{flushleft}
\end{document}
